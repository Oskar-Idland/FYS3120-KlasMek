\documentclass{article}
\usepackage{titlesec}
\usepackage{amsmath}
\usepackage[mathletters]{ucs}
\usepackage[utf8x]{inputenc}
\usepackage[margin=1.5in]{geometry}
\usepackage{enumerate}
\newtheorem{theorem}{Theorem}
\usepackage[dvipsnames]{xcolor}
\usepackage{pgfplots}
\pgfplotsset{compat=1.18}
\setlength{\parindent}{0cm}
\usepackage{graphics}
\usepackage{graphicx} % Required for including images
\usepackage{subcaption}
\usepackage{bigintcalc}
\usepackage{pythonhighlight} %for pythonkode \begin{python}   \end{python}
\usepackage{appendix}
\usepackage{arydshln}
\usepackage{physics}
\usepackage{booktabs} 
\usepackage{adjustbox}
\usepackage{mdframed}
\usepackage{relsize}
\usepackage{physics}
\usepackage[thinc]{esdiff}
\usepackage{esint}  %for lukket-linje-integral
\usepackage{xfrac} %for sfrac
\usepackage{hyperref} %for linker, må ha med hypersetup
\usepackage[noabbrev, nameinlink]{cleveref} % to be loaded after hyperref
\usepackage{amssymb} %\mathbb{R} for reelle tall, \mathcal{B} for "matte"-font
\usepackage{listings} %for kode/lstlisting
\usepackage{verbatim}
\usepackage{graphicx,wrapfig,lipsum,caption} %for wrapping av bilder
\usepackage{mathtools} %for \abs{x}
\usepackage[english]{babel}
\usepackage{cancel}
\definecolor{codegreen}{rgb}{0,0.6,0}
\definecolor{codegray}{rgb}{0.5,0.5,0.5}
\definecolor{codepurple}{rgb}{0.58,0,0.82}
\definecolor{backcolour}{rgb}{0.95,0.95,0.92}
\lstdefinestyle{mystyle}{
    backgroundcolor=\color{backcolour},   
    commentstyle=\color{codegreen},
    keywordstyle=\color{magenta},
    numberstyle=\tiny\color{codegray},
    stringstyle=\color{codepurple},
    basicstyle=\ttfamily\footnotesize,
    breakatwhitespace=false,         
    breaklines=true,                 
    captionpos=b,                    
    keepspaces=true,                 
    numbers=left,                    
    numbersep=5pt,                  
    showspaces=false,                
    showstringspaces=false,
    showtabs=false,                  
    tabsize=2
}

\lstset{style=mystyle}
\author{Oskar Idland}
\title{FYS3120: Classical Mechanics and Electrodynamics \\ Lecture Notes}
\date{}
\includeonly{}

\begin{document}
\maketitle
\tableofcontents
\newpage 

\chapter{Lecture 1}
\section{The Meaning of "Classical"}
Meaning non-quantum. 
\subsection{Analytical Mechanics}
\begin{itemize}
    \item Abstraction of Newton's mechanics, where everything is vectors
    \item Reduces a vector problem to an energy problem
\end{itemize}

\subsection{Lagrange - Hamiltonian formalism}
\paragraph{Lagrangian}
\[
L = K - V
\]
\paragraph{Hamiltonian}
\[
H = K + V
\]
\begin{itemize}
    \item Algorithmic methods for solving problems
    \begin{enumerate}
        \item Find the generalized coordinates
        \item Express the kinetic energy in terms of the generalized coordinates
        \item Derive equations of motion (E.O.M) by a recipe
        \item Solve the differential equations. 
    \end{enumerate}
\end{itemize}

\subsection{Find the Generalized Coordinates}
\subsubsection{Example: Pendulum in a Plane}
\begin{itemize}
    \item Has length $l$ and mass $m$
    \item $x/y$-coordinates are not the best choice. 
    \item Newton: $m \vec{a} = m \vec{g} + \vec{↓} $ and $\left|\vec{r}\right| = l$ with $\vec{r} = (x,y)$
\end{itemize}
\paragraph{Step 1: Find the generalized coordinates}
.\newline
There is a \textbf{constraint equation}:
\[
f(\vec{r}) = \left|\vec{r}\right| - l = 0
\]
The rope has a \textbf{constraint force} $\vec{↓}$, which makes the equation true. We can rewrite $\vec{r}$ using a single coordinate $θ$. 
\[
\vec{r}  = (x,y) = (l \sin \theta, -l \cos \theta)
\]
\paragraph{Step 2: Find the Kinetic Energy}
.\newline
This lets us define the kinetic energy as follows:
\[
K = \frac{1}{2}mv^2 = \frac{1}{2}m(\dot{x} ^2 + \dot{y}^2)
\]
\[
K = \frac{1}{2}m \left((l\cos θ ⋅ \dot{θ})^2 + (l\sin θ ⋅ \dot{θ})^2\right) = \frac{1}{2}ml^2 \dot{θ}^2
\]
\[
V = mgy = -mgl \cos θ
\]

\paragraph{Generalized Coordinates:}
Remaining coordinates (degrees of freedom) after the constraint equation has been applied. In a 3D system with N bodies with coordinates $\vec{r}_i$ where $i ∈ [1,N]$, with $M$ \textbf{holonomic constraints} (velocity independent). 
\[
f_j(\vec{r}_1, \vec{r}_2, ..., \vec{r}_N, t) = 0 \quad j ∈ [1,M]
\]
The degrees of freedom are $d = 3N - M$ where the number 3 comes from the 3 dimensions. 

The generalized coordinates are written as $q_k$ with $k ∈ [1,2, \ldots ,d]$, and the Cartesian coordinates written as:
\[
\vec{r}_i = \vec{r}_i(q_1, q_2, \ldots , q_d, t) \quad | \quad \vec{r}_i = \vec{r}_i(q,t) \quad , q = \left\{q_1, q_2, \ldots  q_d\right\}
\]
\textbf{Notice how the generalized coordinates are not vectors, but the Cartesian coordinates are.}

\paragraph{Configuration Space}
The generalized coordinates $q = \left\{q_1, q_2, \ldots  q_d\right\}$ of a $d$-dimensional space (manifold) called \textbf{configuration space}, With $N$ bodies which position are given by the $3N$-dimenstional vector $\vec{R}= (\vec{r}_1, \vec{r}_2, \ldots  \vec{r}_{N})$. In generalized coordinates $\vec{R} = \vec{R}(q_1, q_2, \ldots  q_d, t)$. This is a hyper-surface of dimension $d = 3N - M$ in the $3N$-dimensional space with $q = \left\{q_1, q_2, \ldots  q_d\right\}$.

\paragraph{Time independent constraints}
\begin{itemize}
    \item Time independent constraint equations
    \item The surface is fixed
    \item The time evolution of $\vec{R}$ is a curve on the surface $\vec{R}(t) = \vec{R}(q(t))$ and $\vec{V} = \vec{R} = \frac{\mathrm{d}\vec{R}}{\mathrm{d}t}$
    \item At a fixed point in time, moving the coordinates $q$ a small amount gives a vector tangent to the surface. 
\end{itemize}

The Cartesian coordinates are dependent on the general coordinates which are time dependent. This is solved by differential equations. 
 %#  1
\input{W03-2} %#  2
\input{W04-1} %#  3
\input{W04-2} %#  4
\input{W05-1} %#  5
\input{W05-2} %#  6
\input{W06-1} %#  7
\input{W06-2} %#  8
\input{W07-1} %#  9
\input{W07-2} %# 10
\input{W08-1} %# 11
\input{W08-2} %# 12
\input{W09-1} %# 13
\input{W09-2} %# 14
\input{W10-1} %# 14
\input{W10-2} %# 16
\input{W11-1} %# 17
\input{W11-2} %# 18
\input{W12-1} %# 19
\input{W12-2} %# 20
\input{W13-1} %# 21
\input{W13-2} %# 22
\input{W14-1} %# 23
\input{W14-2} %# 24
\input{W15-1} %# 25
\input{W15-2} %# 26
\input{W16-1} %# 27
\input{W16-2} %# 28
\input{W17-1} %# 29
\input{W17-2} %# 30
\input{W18-1} %# 31
\input{W18-2} %# 32
\input{W19-1} %# 33
\input{W19-2} %# 34
\input{W20-1} %# 35
\input{W20-2} %# 36
\input{W21-1} %# 37
\input{W21-2} %# 38



\end{document}